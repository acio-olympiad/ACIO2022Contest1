\documentclass{article}
\usepackage{multicol}
\usepackage[utf8]{inputenc}
\usepackage[textwidth=460pt, voffset=0pt]{geometry}
\usepackage{fancyhdr}
\begin{document}
\title{\vspace{-5ex}Bessie the Beef}
\author{\vspace{-5ex}}
\date{\vspace{-5ex}}
\pagestyle{fancy}
\fancyhf{}
\lhead{ACIO 2022 Contest 1}
\rfoot{Page \thepage}

\begin{center}
\huge{Range Messaging}\small\\
\vspace{5ex}
\begin{tabular}{|c|c|}
\hline
Time Limit & Memory Limit \\
\hline
1 second & 512 MB \\

\hline
\end{tabular}
\end{center}
\section*{Statement}
Your friend Sid is a teacher at a local primary school and Lia is his assistant. His class consists of $N$
students who are sitting in a line and are numbered $1$ to $N$. \\

Sid has a very important message to convey to the students in his class. He knows that he can’t an-
nounce the message to the class as a whole and needs to individually tell the message to each student so
that they properly understand. Luckily for Sid, If one student receives the message, they will convey the
message to all of their friends, who will further convey the message to all of their friends, and so on. \\

While Sid was trying to control the class, Lia had observed $M$ friendships between the students. If
student $u$ and $v$ are friends, we can say that a friendship exists between them. \\

Sid, being lazy, wants to only tell the message to students in a continuous range $[L, R]$, $L$ being the
first student he tells the message to, and $R$ being the last student he tells the message to. A range $[L, R]$
is said to be valid if the message will be conveyed to all the students in the class after Sid tells the message
to students in the range $[L, R]$. \\

What is the size of the smallest valid range?

\section*{Input}
The first line contains two integers: $N$ and $M$, the number of students in Sid’s class and the number of friendships observed by Lia. \\

The following $M$ lines contain two integers: $u_i$ and $v_i$, stating a friendship exists between the students numbered $u_i$ and $v_i$. All pairs of $u_i$ and $v_i$ are unique.


\section*{Output}
Output a single integer: The size of the smallest valid range.

\begin{multicols}{2}
\section*{Sample Input 1}
{\tt
6 3\\
2 6\\
1 2\\
5 4}
\section*{Sample Input 2}
{\tt
3 1\\
1 3}
\columnbreak
\section*{Sample Output 1}
{\tt
3
}
\\\\
\section*{Sample Output 2}
{\tt
2
}
\end{multicols}

\section*{Explanation}
\begin{itemize}
\item For Sample Input 1, we can achieve a valid range of size 3 with $L = 2$ and $R = 4$. After Sid tells the
message to students numbered $[2,4]$, student 2 will convey the message to student 1 and student 6,
and student 4 will convey the message to student 5. A valid range of size less than 3 is not possible.
\end{itemize}

\section*{Constraints}
\begin{itemize}
\item $1 \le N \le 10^5$
\item $0 \le M \le \mathrm{min}(N \times (N - 1)/2, 10^5)$
\end{itemize}

\section*{Subtasks}
\begin{tabular}{l*{6}{c}r}
Number & Points & Other constraints\\
\hline
1 & 10 & $M = 0$ \\
2 & 17 & $N \le 100$ \\
3 & 25 & $N \ge 3$ (see below) \\
4 & 15 & $N \le 1000$ \\
5 & 33 & No further constraints.
\end{tabular} \\

Additionally, for Subtask 3, each student (excluding $1$, $2$ and $N$) has a friendship with exactly one of student $1$, student $2$ or student $N$, either directly or indirectly.
\end{document}
