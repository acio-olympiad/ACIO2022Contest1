\documentclass{article}
\usepackage{multicol}
\usepackage[utf8]{inputenc}
\usepackage[textwidth=460pt, voffset=0pt]{geometry}
\usepackage{fancyhdr}
\begin{document}
\title{\vspace{-5ex}Squirrel Postal Service}
\author{\vspace{-5ex}}
\date{\vspace{-5ex}}
\pagestyle{fancy}
\fancyhf{}
\lhead{ACIO 2022 Contest 1}
\rfoot{Page \thepage}

\begin{center}
\huge{Squirrel Postal Service}\small\\
\vspace{5ex}
\begin{tabular}{|c|c|}
\hline
Time Limit & Memory Limit \\
\hline
1 second & 512 MB \\

\hline
\end{tabular}
\end{center}
\section*{Statement}
Lisa, the operations manager for the Squirrel Postal Service, needs your help! \\

The Squirrel Postal Service delivers items across the whole town, which can be
described as $N$ huts (numbered $1$ through $N$) with $N-1$ bidirectional roads connecting them, such that
you can reach any hut from any other hut by travelling along one or more of the
roads. \\

The Squirrel Postal Service has $N$ deliveries today, each going to a different
hut. A route is any sequence of huts such that each adjacent
pair of huts in the sequence is connected by a road. As such, Lisa wants to find length of the shortest
route that starts at the main depot (hut $1$) and passes through every
hut in town. \\

Can you help Lisa by outputting the number of huts in the shortest route starting at hut 1
and passing through all other huts?

\section*{Input}
The first line contains the integer $N$. \\

The next $N-1$ lines contain two integers each, representing a pair of huts connected
by road.

\section*{Output}
Output one integer, the number of huts in the shortest route as described in the Statement section.

\begin{multicols}{2}
\section*{Sample Input 1}
{\tt
6\\
1 3\\
1 6\\
4 3\\
2 5\\
3 2}
\columnbreak
\section*{Sample Output 1}
{\tt
8
}
\end{multicols}
\newpage
\begin{multicols}{2}
\section*{Sample Input 2}
{\tt
5\\
5 1\\
1 2\\
2 3\\
3 4
}
\columnbreak
\section*{Sample Output 2}
{\tt
6
}
\end{multicols}

\section*{Explanation}
\begin{itemize}
\item For the first example, the shortest route is 1, 6, 1, 3, 4, 3, 2, 5.
\item For the second example, the shortest route is 1, 5, 1, 2, 3, 4.
\end{itemize}

\section*{Constraints}
\begin{itemize}
\item $1 \le N \le 10^5$
\end{itemize}

\section*{Subtasks}
\begin{tabular}{l*{6}{c}r}
Number & Points & Other constraints\\
\hline
1 & 35 & $N \le 1000$ \\
2 & 50 & Each hut is connected to at most 2 other huts. That is, the town is a line. \\
3 & 15 & No further constraints.
\end{tabular}
\end{document}
